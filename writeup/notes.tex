\documentclass[a4paper]{article}
\setlength\parindent{0pt}

\usepackage{amsthm, amsmath, amssymb, verbatim, enumerate, mathtools,
algorithm, pgfplots}
\usepackage{rotating}
\usepackage[margin=0.5in]{geometry}
\usetikzlibrary{arrows, quotes}

\begin{document}

Relation Web
\begin{itemize}
    \item Let $D$ be the set of descriptors.
    \item Let $R$ be the set of possible relations.
    \item Let $G = \left( V, E \right)$ be a directed graph where $V = D$ and
        $E$ is a set of edges such that each $v \in V$ has $|R| \times |D|$
        edges, one per relation per descriptor (exhaustively listing all 
        possible relations between this descriptor and all other descriptors).
    \item Goal: assign each edge $e \in E$ a weight $P(e) \in [0, 1]$ that
        represents the probability of that being a valid relation.
    \item Idea: as triples are processed, adjust the weight of not only the 
        ``explicit'' relation between the relevant descriptors that is directly
        stated with that triple, but also the ``implicit'' relation(s).
        % Knowledge graph?
\end{itemize}<++>

%Probability Representation
%\begin{itemize}
%    \item Let the set $R$ contain all the relations between all possible
%        relations between all descriptors. 
%    \item Sample Space: Assign each $r \in R$ a value $0, 1$ indicating whether
%        or not that relation is valid. Let $S$ be the space of all such
%        outcomes.
%    \item Let $T$ be the set of all training triples.
%    \item Want to find: $s \in S$ such that $P(s | T)$ is maximized.
%    \item Assume perfect relation extraction from triples.
%    \item Let $R_E$ be the set of all explicit relations extracted from $T$.
%    \item Let $R_I$ be the set of all explicit relations extracted from $R_E$.
%\end{itemize}

Rudimentary Algorithm
\begin{itemize}
%    \item Assumptions:
%    \begin{itemize}
%        \item Triple classification is perfect.
%    \end{itemize}
    \item Explicit relation detection:
    \begin{itemize}
        \item For each triple, label the detected explicit relations with their
            probabilities (e.g., logistic regression confidence measure).
    \end{itemize}
    \item Drawing implications:
    \begin{itemize}
        \item For every pair of descriptors $A, B$ find all non-zero paths $A
            \rightarrow B$. Store these paths and their vertex labels and edge
            probabilites in a data structure indexed by descriptor pair.
        \item ``Collate'' the entries for every pair of descriptors of the same
            respective labels, finding clusters of paths that tend to be
            correlated with one another by finding correlation scores for each
            pair of paths.
        \item When processing a new triple:
        \begin{itemize}
            \item Update explicitly identified edge(s) with confidence values.
            \item For each path this edge is a part of, update all correlated
                paths by an amount proportional to this path's edge weights and
                the correlation score.
        \end{itemize}
    \end{itemize}
\end{itemize}

%\begin{tikzpicture}
%
%\tikzset{vertex/.style = {shape=circle,draw,minimum size=1.5em}}
%\tikzset{edge/.style = {->,> = latex'}}
%% vertices
%\node[vertex] (A) at  (0,5) {$A$};
%\node[vertex] (X) at  (0,0) {$X$};
%\node[vertex] (B) at  (5,0) {$B$};
%%edges
%%\draw[edge] (A) to[bend left=50] (X) node [midway, fill=white] {1} ;
%%\draw[edge] (A) to["2", bend left] (X);
%%\draw[edge] (X) to["1", bend right=10] (A);
%%\draw[edge] (X) to["2", bend left=10] (A);
%%\draw[edge] (A) to["1", bend right] (B);
%%\draw[edge] (A) to["2", bend left] (B);
%%\draw[edge] (B) to["1", bend right=10] (A);
%%\draw[edge] (B) to["2", bend left=10] (A);
%%\draw[edge] (B) to["1", bend right] (X);
%%\draw[edge] (B) to["2", bend left] (X);
%%\draw[edge] (X) to["1", bend right=10] (B);
%%\draw[edge] (X) to["2", bend left=10] (B);
%
%\end{tikzpicture}
\end{document}
